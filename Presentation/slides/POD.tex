\begin{frame}{Proper Orthogonal Decomposition}
	\Large{\highlightB{POD basis generation}}\\ \normalsize
	The POD is a method to find the optimal orthogonal space in a linear manner.

	
	\begin{itemize}
		\item Set based approach $\Xi_{train} = \left\{ \mu_{1},..., \mu_{n_{train}} \right\}$
		\item $\mathbb{X}_{n_{train}} = \text{span}\left\{ u^\mathcal{ N }(\mu) | \mu \in \Xi_{train} \right\}$
	\end{itemize}

	\Large{\highlightB{Eigen value problem (EVP)}} \normalsize
	\begin{align*}
	C(v) =
	\frac{1}{n_{train}}
	\sum_{m=1}^{n_{train}} 
	\left( v ,u^\mathcal{N}(\mu_{m}) \right)_\mathbb{X} u^\mathcal{N}(\mu_{m}) 
	, \quad 
	v \in \mathbb{X}_{n_{train}}
	\\
	\left(C(\xi_n),u^\mathcal{N}(\mu_{m})\right)_\mathbb{X}
	=
	\lambda_n \left(\xi_n,u^\mathcal{N}(\mu_{m}) \right)_\mathbb{X}
	, \quad 
	1 \leq m \leq n_{train}
	\end{align*}

	\Large{\highlightB{POD error}} \normalsize
	\begin{align*}
	\frac{1}{n_{train}}
	\sum_{ \mu \in \Xi_{train} }   
	\inf_{ w_N \in \mathbb{X}_{n_{train}} } \| u^\mathcal{N}(\mu) - w_N \|_\mathbb{X}^2 
	=
	\sum_{n=N+1}^{n_{train}}\lambda_{n}
	\end{align*}
	
	\begin{itemize}
		\item Truncate all modes $> N$
	\end{itemize}

	\end{frame}


\begin{frame}{Proper Orthogonal Decomposition}
	\Large{\highlightB{Snapshot Matrix}}
	\normalsize
	
	\begin{align*}
	\mathcal{W} = 
	\begin{bmatrix} 
	\vline & \vline & & \vline\\ 
	\underline{u}^{\mathcal{ N }}(\mu_{1}) & \underline{u}^{\mathcal{ N }}(\mu_{2}) & \hdots & \underline{u}^{\mathcal{ N }}(\mu_{n_{train}})  \\
	\vline & \vline & & \vline 
	\end{bmatrix}
	\end{align*}
	
	Truth solution
	\begin{align*}
	u^{\mathcal{ N }}(\mu) = \sum_{i=1}^{\mathcal{ N }} u_{i}^{\mathcal{ N }}(\mu) \varphi_i, 
	\qquad
	\underline{u}^{\mathcal{ N }}(\mu) = 
	\begin{pmatrix}
	u_{1}^{\mathcal{ N }}(\mu)\\  \vdots \\ u_{\mathcal{ N }}^{\mathcal{ N }}(\mu)
	\end{pmatrix}
	\end{align*}
	
	
\end{frame}

\begin{frame}{Proper Orthogonal Decomposition}
	\Large{\highlightB{Structure of the snapshot matrix}}
	\normalsize

	\begin{minipage}[t]{0.3\textwidth}
		\scalebox{0.75}{
			\tikzset{
	main/.style={black, line width=0.4mm, opacity=1},
	second/.style={gray, opacity=5},
	arrow/.style={-latex, shorten >=1ex, shorten <=1ex, bend angle=45}
}
\begin{tikzpicture}
\draw[main] (0,0) -- (2,0) -- (2,8)-- (0,8)-- (0,0);

\draw[second] (0,2) -- (2,2) ;
\draw[second] (0,4) -- (2,4) ;
\draw[second] (0,6) -- (2,6) ;


\draw[decorate, decoration={brace,mirror}, yshift=.5ex]  (2,8) -- node[above=0.4ex] {$n$}  (0,8);
\draw[decorate, decoration={brace}, xshift=-.5ex]  (0,0) -- node[left=0.4ex] {$m$}  (0,8);

\draw (1,7) node {$w_1$};
\draw (1,5) node {$w_2$};
\draw (1,3) node {$\vdots$};
\draw (1,1) node {$w_{np}$};


\end{tikzpicture}
		}
	\end{minipage}
	\hfill
	\begin{minipage}[t]{0.4\textwidth}\vspace{-60mm}
	\highlight{Partitioning} cases
	\begin{itemize}
		\item Domain decomposition
		\item Manually blocked
	\end{itemize}
	\end{minipage}
\end{frame}
%%%%%%%%%%%%%%%%%%%%%%%%%%%%%%%%%%%%%%%%%%%%%%%%%%%%%%%%%%%%%%%%%%%%%%%%%%%%%%%%
\begin{frame}{Proper Orthogonal Decomposition}

	\begin{minipage}[t]{0.48\linewidth}
		\Large{\highlightB{EVP}}
		\normalsize
		
		\begin{itemize}
			\item Solving eigen value problem (EVP) of the snapshot matrix
			\begin{align*}
				\mathcal{W}^T \mathcal{W} = V D V^T\\
				U = \mathcal{W}VD^{-\frac{1}{2}}
			\end{align*}
		\end{itemize}
	\end{minipage}
	\hfill%
	\begin{minipage}[t]{0.48\linewidth}
		\Large{\highlightB{SVD}}
		\normalsize
		
		\begin{itemize}
			\item Singular Value Decomposition (SVD) of the snapshot matrix
			\begin{align*}
				\mathcal{W} = USV^T
			\end{align*}
		\end{itemize}	 
	\end{minipage}
	
	\vspace{10mm}
  	
	Relation singular values and eigen values
	\begin{align*}   
		\mathcal{W}^T \mathcal{W} &= (USV^T)^T USV^T = VSU^T USV^T = V S^2 V^T
	\end{align*}
	The left singular are computed with 
	\begin{align*}
		U = \mathcal{W}VS^{-1}
	\end{align*}


	
\end{frame}

%%%%%%%%%%%%%%%%%%%%%%%%%%%%%%%%%%%%%%%%%%%%%%%%%%%%%%%%%%%%%%%%%%%%%%%%%%%%%%%%
\begin{frame}{Proper Orthogonal Decomposition}
	\Large{\highlightB{Parallel SVD}}
	\normalsize approach introduced by Beattie, Borggaard and Iliescu in 2007
	
	\vspace{0.25cm}
	Singular Value Decomposition (SVD) of the snapshot Matrix $W$
	\begin{align*}
	\mathcal{W} &= U S V^{T} 
	\end{align*}
	$U$ are the POD modes.
	\vspace{0.25cm}
	
	Compute SVD of the local snapshot matrices $ \mathcal{W} = [W_1^T ... W_{n_p}^T]^T	$
	\begin{align*}
	\text{svd}(W_k) &= [ U,S,\boldsymbol{V_k}]
	\end{align*}
	Compress local right singular vectors $\boldsymbol{V_k}$ using SVD
	\begin{align*}
	\hat{ \mathcal{V} } = [V_1 ... V_{n_p}]
	\qquad
	\text{svd}( \hat{ \mathcal{V} }) = 
	[\mathcal{T},\mathcal{M},\boldsymbol{\mathcal{V}}]
	\end{align*}
	Compute left singular vectors	
	\begin{align*}
	\text{svd}( \mathcal{ W }\mathcal{V} ) =
	[\boldsymbol{\mathcal{U}}, \mathcal{S}, \mathcal{Z}]
	\end{align*}

	
\end{frame}

